\documentclass[a4paper, 10pt]{article}

\usepackage{bjerkingcv}

\newcommand{\unit}{Life Science} % NAMN PÅ ENHET
\newcommand{\unitdetail}{Process Engineering \& Project Management} % DETALJER OM PERSONENS ENHETSROLL
\newcommand{\name}{Philip Tunbjer} % NAMN PÅ ANSTÄLLD
\newcommand{\role}{Senior Consultant Life Science} % NAMN PÅ ANSTÄLLD
\newcommand{\phone}{010-211 85 67} % TEL TILL ANSTÄLLD
\newcommand{\email}{philip.tunbjer@bjerking.se} % NAMN PÅ ANSTÄLLD
\newcommand{\employmentyear}{2015} % ÅRET/DATUMET DÅ PERSON ANSTÄLLDES
\newcommand{\presentationtext}{} % KORT PRESENTATIONSTEXT


\newcommand{\portrait}{img/portrait.png} % PATH TO PROFILE PICTURE
\renewcommand{\dateseparator}{--} % ÄNDRA EJ.

\begin{document}
\thispagestyle{presentation}
\pagestyle{plain}
\noindent
\begin{minipage}[t]{0.3\textwidth}
    \strut\vspace*{-\baselineskip}\newline
    \begin{tabular*}{1\linewidth}{p{1\linewidth}}
        {\LARGE Curriculum vitae}\\
        \includegraphics[width=120pt, height=120pt]{\portrait} \\
        \textbf{\name} \\
        \role{} \\
        \unitdetail{} \\
        \\
        \textbf{Kontakt} \\
        \phone{} \\
        \href{mailto:\email}{\email} \\
    \end{tabular*}
\end{minipage}%
\hfill
\begin{minipage}[t]{0.6\textwidth}
    \strut\vspace*{-\baselineskip}\newline
    \begin{tabular*}{1\linewidth}{p{0.7\linewidth} r}
        {\Large Presentation} & \\
        \mc{2}{\presentationtext} \\
        \textbf{Anställdes vid bjerking} & \\
        \employmentyear & \\
        & \\
        \textbf{Tidigare anställningar}& \\
        ÅF Industry AB & 2015 \\
        PRC Engineering AB & 2014-2015 \\
        Axis Communications AB & 2013 \\
        Fagerdala World Foams AB & 2008 \\
        Fagerdala Marine Systems AB & 2006 \\
        & \\
        \textbf{Utbildning} & \\
        Civilingenjör i Maskinteknik & 2007-2013 \\
        Webutveckling I & 2021 \\
        Algoritmer och Datastrukturer & 2020 \\
        Objektorienterad Programmering & 2020 \\
        Beräkningsvetenskap I & 2019 \\
        Matematik Alfa & 2006 \\
        & \\
        \textbf{Kurser}& \\
        BAS-P \& BAS-U grundutbildning & 2019 \\
        CE-märkning av medicintekniska produkter & 2015 \\
        Plant 3D grundutbildning & 2014 \\
        GMP grundutbildning & 2014 \\
    \end{tabular*}
\end{minipage}
\mbox{ }
\begin{longtable}{@{\extracolsep{\fill}}p{0.2\textwidth} p{0.65\textwidth} Rp{0.1\textwidth}}
    {\large Referensobjekt} & {\large Omfattning} & {\large Tid} \\
    \toprule
    \endfirsthead

    {\large Referensobjekt} & {\large Omfattning} & {\large Tid} \\
    \toprule
    \endhead

    \bottomrule
    \endlastfoot

    \textbf{Lokaleffektivisering}

    Cytiva Sweden AB

    \textit{Roll: Uppdragsledare}
    &
    \textit{Uppdragsbeskrivning}: Effektivisering av befintliga lokaler inom Cytivas R\&D-avdelning.

    \textit{Omfattning}: Fem byggnader vari allt från kontor och lab till WC och förråd ses över. 

    \textit{Roller}: 
    \begin{itemize}
        \item Uppdragsledning
        \item Logistik
    \end{itemize}
    \textit{Typ av uppdrag}: Förstudie.

    \textit{Organisationskomplexitet}: R\&D-avdelning med ca 440 personer, uppdelade i ca 18 sektioner i olika discipliner.

    \textit{Uppdragsvolym}: Totalt ca 1100 timmar.

    \textit{Genomfört arbete}: 
    \begin{itemize}
        \item I första hand leda framdriften av uppdraget och dess medlemmar (Arkitekter, E, V/VVS och K), hantera tidplan, ekonomi och samband mellan kund och projekt.
        \item Genomföra intervjuer med samtliga sektioner inom R\&D för att ta reda på behov och brister.
        \item Författa rapporter och presentera detta inför styrgrupp efter varje steg i uppdraget.
    \end{itemize}
    Pågående uppdrag: CV uppdateras efterhand. Avslut beräknas till och med mars 2021.
    &
    2020-nu
    \\ \midrule
    \textbf{Chopin}

    Locum AB

    \textit{Roll: Projektingenjör}
    &
    \textit{Uppdragsbeskrivning}: Införskaffande av medicinteknisk utrustning till ny operationsbyggnad i Karolinska Huddinge.

    \textit{Roller och ansvar}: Dokumentationshantering för att underlätta inför överlämningsprocessen från projektet till sjukhuset.
    &
    2020
    \\ \midrule
    \textbf{Gävlegatan 55}

    Stockholms läns landsting

    \textit{Roll: Projektledning}
    &
    \textit{Uppdragsbeskrivning}:

    \textit{Omfattning}:
    \begin{itemize}
        \item Anskaffning av ca 820 utrustningar och produkter till ombyggd del av gamla Karolinska sjukhuset.
        \item Medicinteknisk utrustning, bland annat PAMS (patientmonitoreringssystem), steriltekniska produkter, undersökningslampor, defibrillatorer, akutvagnar, försörjningsenheter, sugejektorer och infusionspumpar.
        \item Arbetsstolar och skrivbord.
    \end{itemize}
    \textit{Roller och ansvar}:
    \begin{itemize}
        \item Projektleda anskaffning av medicintekniska och övriga produkter.
        \item Leda förnyad konkurrensutsättning för ej avtalade produkter.
        \item Leverans- och installationssamordning.
        \item Koordinera driftsättning och överlämning av angivna produkter.
    \end{itemize}
    &
    2019
    \\ \midrule
    \textbf{Nya Karolinska Sjukhuset Solna}

    Stockholms läns landsting

    \textit{Roll: Konstruktör / Projekteringsingenjör / Projektledning}
    &
    \textit{Uppdragsbeskrivning}: Projektledning, projektering och installationssamordning av medicintekniska produkter på nya Karolinska Sjukhuset inom MT Programmet.

    \textit{Omfattning}:
    \begin{itemize}
        \item Takhängd byggnadspåverkande medicinteknisk utrustning (THU), Ca. 1700 utrustningar fördelat på 700 rum.
        \item En hyperbar oxygenbehandlingskammare (HBO).
        \item Endoskopiutrustning till operation (OP).
    \end{itemize}
    \textit{Roller och ansvar}:
    \begin{itemize}
        \item Ritningskoordinator och CAD-stöd för MT Programmet samt leverantörer.
        \item Projektering av upphandlad utrustning.
        \item Avrop av upphandlad utrustning.
        \item Projektledning av hyperbar oxygenbehandlingskammare.
        \item Leverans- och installationssamordning.
        \item Driftsättning och överlämning av medicinteknisk utrustning.
    \end{itemize}
    \textit{Organisationskomplexitet}:
    \begin{itemize}
        \item Flertalet intressenter samt intern organisationsstruktur gav upphov till en omfattande komplexitet: Landsting/Region, Beställare, Entreprenör, Slutanvändare, Drift, Leverantörer, Arkitekter, Utrustningsprojekt.
        \item Kommunikation skedde på svenska och engelska.
    \end{itemize}
    \textit{Genomfört arbete}:
    \begin{itemize}
        \item Ta fram CAD underlag åt olika delprojekt för olika ändamål.
        \item Detaljprojektering av upphandlad utrustning.
        \item Underhålla en utrustningslista, en av grundpelarna i THU-projektet.
        \item Upprätta en databas för utrustningshantering.
        \item Stödja, styra samt förbereda leverantörer under installationsarbetet.
        \item Utforma, genomföra och följa upp montageplatskontroller.
        \item Samordna med byggentreprenör och dess underentreprenörer.
        \item Kontrolldokument för utrustningar för att säkerställa överensstämmelse med kravställning.
        \item Samordna teknisk driftsättning av medicinteknisk utrustning.
        \item Säkerställa relevant teknisk dokumentation.
        \item Stöd i den ekonomiska uppföljningen av projektet.
        \item Kravställa och avropa tillbehör till mediabryggor (ca. 540 unika tillbehör).
        \item Ta fram design till OP-salarnas sladd- och slanghållare.
        \item Kravställning och upphandling av kran för tunga lyft.
        \item Projektledning vid färdigställandet av HBO.
        \item Medverka vid upphandling av viss endoskopiutrustning.
        \item Upprätta och följa upp mötesprotokoll. Dokumenthantering och diarieföring.
        \item Tidplanering.
        \item Genomförande och dokumentering av tekniska utredningar och instruktioner.
        \item Medverka i riskhanteringsprocessen.
    \end{itemize}
    &
    2014-2018
    \\ \midrule
    \textbf{Framtagning P\&ID Pannrum}

    Crucell Sweden AB

    \textit{Roll: Konstruktör}
    &
    \textit{Uppdragsbeskrivnin}: Utfärda ett uppdaterat P\&ID-schema för en del av förtagets anläggning som hanterar ånga och kondensat.

    \textit{Arbetsuppgifter}: Besök på plats för kartläggning samt i Auto CAD schematiskt rita upp rördragningen.
    &
    2014
    \\ \midrule
    \textbf{PARA aktiva avlopp}

    Studsvik Nuclear AB
    
    \textit{Roll: Konstruktör}
    &
    \textit{Uppdragsbeskrivning}: Förnyelse av rördragning i kulvertsystem. 

    \textit{Arbetsuppgifter}: Ta fram en isometri med hjälp av Plant 3D
    &
    2014
    \\ \midrule
    \textbf{Axis Communications AB}

    \textit{Roll: Konstruktör}
    &
    \textit{Uppdragsbeskrivning}: Bygga upp ett bibliotek av standardkomponenter.

    \textit{Arbetsuppgifter}: Med hjälp av CAD-verktyget Creo 2.0 skapa och tilldela komponenter olika attribut i databashanteraren Windchill
    &
    2013
    \\ \midrule
    \textbf{Hexicon AB}

    \textit{Roll: Konstruktör}
    &
    \textit{Uppdragsbeskrivning}: Ta fram en generell design på en flytande vindkraftplattform.

    \textit{Arbetsuppgifter}: Med hjälp av CAD ta fram ritningsunderlag och presentationsmaterial för plattformen. Skapa de första ritningsmallarna och strukturen för den digitala uppbyggnaden av plattformen.
    &
    2010, 2011, 2012
    \\ \midrule
    \textbf{Fagerdala World Foams AB}

    \textit{Roll: Installatör}
    &
    \textit{Uppdragsbeskrivning}: Anlägga en testyta av ett skeppsskrov med skummaterial åt det tyska varvet Lürssen.

    \textit{Arbetsuppgifter}: Rengöring av skrovet, förberedelse för montage, anläggning av lim, fiberglas och skummaterial.
    &
    2008
    \\ \midrule
    \textbf{Fagerdala Marine Systems AB}

    \textit{Roll: Installatör}
    &
    \textit{Uppdragsbeskrivning}: Installera ett teakdäck samt reparera ett äldre på ett fartyg i Karibien.

    \textit{Arbetsuppgifter}: Mäta in och utföra grundarbete på den yta som det nya däcket skulle läggas.
    &
    2006
    \\
\end{longtable}
\end{document}